%% ========================================================================
%%  JANUS-SR   Janus Speech Recognition Toolkit
%%             ------------------------------------------------------------
%%             Object: Description of featDesc
%%             ------------------------------------------------------------
%%
%%  Author  :  Florian Metze & many others
%%  Module  :  featDesc.tex
%%  Date    :  $Id: featDesc.tex 2390 2003-08-14 11:20:32Z fuegen $
%%
%%  Remarks :  
%%
%% ========================================================================
%%
%%   $Log$
%%   Revision 1.2  2003/08/14 11:18:59  fuegen
%%   Merged changes on branch jtk-01-01-15-fms (jaguar -> ibis-013)
%%
%%   Revision 1.1.2.4  2002/11/19 13:23:30  metze
%%   Beautification
%%
%%   Revision 1.1.2.3  2002/11/19 09:17:44  fuegen
%%   minor changes for overfull hboxes
%%
%%   Revision 1.1.2.2  2002/07/31 13:10:12  metze
%%   *** empty log message ***
%%
%%   Revision 1.1.2.1  2002/07/30 13:57:39  metze
%%   *** empty log message ***
%%
%% ========================================================================

\section{\Jlabel{file}{featDesc}}

The feature description file, read by the
\Jref{module}{FeatureSet}. An example looks like this:

\begin{verbatim}
# =======================================================
#  JanusRTk     Janus Recognition Toolkit
#               -----------------------------------------
#               Object: Feature Description
#               -----------------------------------------
#
#  Author    :  Hagen Soltau
#  Module    :  featDesc
#  Remarks   :  based on Hua's new frontend, 40 dimensions
#
#  $Log$
#  Revision 1.2  2003/08/14 11:18:59  fuegen
#  Merged changes on branch jtk-01-01-15-fms (jaguar -> ibis-013)
#
#  Revision 1.1.2.4  2002/11/19 13:23:30  metze
#  Beautification
#
#  Revision 1.1.2.3  2002/11/19 09:17:44  fuegen
#  minor changes for overfull hboxes
#
#  Revision 1.1.2.2  2002/07/31 13:10:12  metze
#  *** empty log message ***
#
#  Revision 1.1.2.1  2002/07/30 13:57:39  metze
#  *** empty log message ***
#
#  Revision 1.1  2002/03/04 16:10:49  soltau
#  Initial revision
#
# =======================================================

global WARPSCALE warpScales meanPath
global WAVFILE OLDSPK sas pms


# -------------------------------------------------------
#  Load Mean Vectors
# -------------------------------------------------------

if {![info exist OLDSPK] || $OLDSPK != $arg(spk) } {

  if {[llength [info command ${fes}Mean]]} { 
    ${fes}Mean  destroy
    ${fes}SMean destroy
  }

  if {[file exist $meanPath/$arg(spk).mean]} {
    FVector ${fes}Mean  13
    FVector ${fes}SMean 13
    writeLog stderr "$fes Loading $meanPath/$arg(spk).mean"
    ${fes}Mean  bload $meanPath/$arg(spk).mean
    ${fes}SMean bload $meanPath/$arg(spk).smean
    set OLDSPK $arg(spk)
  } else {
    writeLog stderr "$fes Loading $meanPath/$arg(spk).mean FAILED"
  }
}


# -------------------------------------------------------
#  Load ADC segment...
# -------------------------------------------------------

if {![info exist WAVFILE] || $WAVFILE != $arg(ADCFILE)} {

  set WAVFILE $arg(ADCFILE)

  if {[file exist $arg(ADCFILE).shn]} {
    $fes    readADC         ADC             $arg(ADCFILE).shn \
      -h 0 -v 0 -offset mean -bm shorten
  } else {
    $fes    readADC         ADC             $arg(ADCFILE)     \
      -h 0 -v 0 -offset mean -bm auto
  }

  $fes spectrum FFT     ADC 20ms
}


# -------------------------------------------------------
#  Get warp
# -------------------------------------------------------

if {![info exist WARPSCALE]} { 
  if [info exist warpScales($arg(spk))] { 
     set WARP $warpScales($arg(spk))
  }  else {
     set WARP 1.00
  } 
} else { set WARP $WARPSCALE } 

writeLog stderr "$fes ADCfile $arg(utt) WARP $WARP"


# -------------------------------------------------------
#  Vocal Tract Length Normalization + MCEP
# -------------------------------------------------------

$fes VTLN     WFFT      FFT  $WARP -mod lin -edge 0.8

if { [llength [objects FBMatrix matrixMEL]] != 1} {
   set melN 30
   set points [$fes:FFT configure -coeffN]
   set rate   [expr 1000 * [$fes:FFT configure -samplingRate]]
   [FBMatrix matrixMEL] mel -N $melN -p $points -rate $rate
}

$fes   filterbank       MEL             WFFT           matrixMEL
$fes   log              lMEL            MEL            1.0 1.0

set cepN 13

if { [llength [objects FMatrix matrixCOS]] != 1} {
   set n [$fes:lMEL configure -coeffN]
   [FMatrix matrixCOS] cosine $cepN $n -type 1
}

$fes   matmul           MCEP            lMEL            matrixCOS


# -------------------------------------------------------
#  Mean Subtraction, Delta, Delta-Delta and LDA
# -------------------------------------------------------

$fes meansub  FEAT    MCEP  -a 2 -mean ${fes}Mean -smean ${fes}SMean
$fes adjacent FEAT+   FEAT  -delta 5

if { [$fes index LDAMatrix] > -1} {
    $fes matmul   LDA     FEAT+ $fes:LDAMatrix.data -cut 32
}

if [info exists pms] {
    foreach p [$pms] {
        $fes matmul OFS-$p LDA $pms:$p.item(0)
        if [info exists sas] {
            $sas adapt $fes:OFS-$p.data $fes:OFS-$p.data 0
        }
    }
} else {
    if [info exists sas] {
        $sas adapt $fes:LDA.data $fes:LDA.data 0
    }
}

\end{verbatim}

Errors in the featDesc are not always easy to track. A much-used
strategy to debug errors in the featDesc is to plaster it with
\texttt{puts ``I am here''} commands, to find out where exactly in the
code the offending operation occurs.


%%% Local Variables: 
%%% mode: latex
%%% TeX-master: t
%%% End: 
