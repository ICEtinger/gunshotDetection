%% ========================================================================
%%  JANUS-SR   Janus Speech Recognition Toolkit
%%             ------------------------------------------------------------
%%             Object: Description of a context free grammar
%%             ------------------------------------------------------------
%%
%%  Author  :  Christian Fuegen & many others
%%  Module  :  cfg.tex
%%  Date    :  $Id: cfg.tex 2390 2003-08-14 11:20:32Z fuegen $
%%
%%  Remarks :  
%%
%% ========================================================================
%%
%%   $Log$
%%   Revision 1.2  2003/08/14 11:18:57  fuegen
%%   Merged changes on branch jtk-01-01-15-fms (jaguar -> ibis-013)
%%
%%   Revision 1.1.2.2  2003/04/30 15:44:01  metze
%%   Changes before the new repository
%%
%%   Revision 1.1.2.1  2002/11/19 08:38:38  fuegen
%%   initial version
%%
%%
%% ========================================================================

\section{\Jlabel{file}{ContextFreeGrammars}}

This section describes our internal context free grammar format,
called SOUP-Format. We are usually using semantic instead of syntactic
context free grammars. They are read by \Jref{module}{CFGSet}. A not
completely specified example looks like:

\begin{verbatim}
# ======================================================================
# example grammar
# ======================================================================

# ----------------------------------------
# request path description
#       how do i
#       i want to find the way
#       can you take me
# ----------------------------------------
s[request-path-description]
        ( *PLEASE [_NT_how-to-go]    [obj_desc]     *PLEASE )
        ( *PLEASE [_NT_find-the-way] [obj_desc]     *PLEASE )
        ( *PLEASE [_NT_take-me]      [obj_desc]     *PLEASE )
        ( *PLEASE [_NT_how-about]    [obj_desc]     *PLEASE )
        ( *PLEASE [_NT_how-to-find]  [_NT_obj_desc] *PLEASE )

[_NT_how-to-go]
        ( how do i                           GO )
        ( [_NT_can-you-show|tell] *me how to GO )
        ( i WANT                          to GO )
        ( i NEED                          to GO )

[_NT_can-you-show|tell]
        ( *CAN_YOU SHOW )
        ( *CAN_YOU TELL )


[obj_desc]
        (                     to [_NT_obj_desc] )
        ( from [_NT_obj_desc] to [_NT_obj_desc] )

[_NT_obj_desc]
        ( *the          [objnm] )
        (  the *NEAREST [objcl] )
        (  A            [objcl] *NEARBY )

[objcl]
        ( [objcl_bakery] )
        ( [objcl_bank] )

[objcl_bakery]
        ( bakery )

CAN
        ( can )
        ( could )

CAN_YOU
        ( CAN you )

SHOW
        ( show )
        ( display )

TELL
        ( tell )
        ( explain *to )

GO
        ( get )
        ( go )

# ----------------------------------------
# greeting / farewell
#       hello
#       good bye
#       bye bye
# ----------------------------------------
s[greeting]
        ( [_NT_greeting] )

[_NT_greeting]
        ( hello )
        ( hi )

s[farewell]
        ( [_NT_farewell] )

[_NT_farewell]
        ( *good +bye )
\end{verbatim}

Non terminal symbols could either be surrounded by \texttt{[]} or could be
started with a capital letter. Terminal symbols have to be started
with a lower case letter. If you start a non terminal with a capital
our with the modifier \texttt{\_NT\_}, it is classified as an auxilliary non
terminal and will per default not occur in the parse tree. To express
optionality of a terminal or non terminal you have to use \texttt{*} and to
express repeatability you have to use \texttt{+} in front of a symbol. It is
also posible to combine optionality and repeatability by using \texttt{*+}.

Rules consist of a left hand side (LHS, the head of the rule) and a
right hand side (RHS, the body of the rule). If you want to use a rule
also as top level rule, i.e. a rule where you can start to parse from,
you have to put the modifier \texttt{s} in front of the rule. As you
can see above there are three top level rules:
\texttt{[request-path-description]}, \texttt{[greeting]} and
\texttt{[farewell]}. The lines in a RHS of a rule are interpreted as a
disjunction, the terminals and non terminals in one line as a
conjunction. It is not neccessary to define the rules in a special
order.

%%% Local Variables: 
%%% mode: latex
%%% TeX-master: t
%%% End: 
