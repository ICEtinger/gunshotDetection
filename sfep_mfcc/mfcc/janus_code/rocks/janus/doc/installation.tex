%% ========================================================================
%%  JANUS-SR   Janus Speech Recognition Toolkit
%%             ------------------------------------------------------------
%%             Object: Documentation for Installation
%%             ------------------------------------------------------------
%%
%%  Author  :  Florian Metze & many others
%%  Module  :  installation.tex
%%  Date    :  $Id: installation.tex 2390 2003-08-14 11:20:32Z fuegen $
%%
%%  Remarks :  
%%
%% ========================================================================
%%
%%   $Log$
%%   Revision 1.2  2003/08/14 11:18:47  fuegen
%%   Merged changes on branch jtk-01-01-15-fms (jaguar -> ibis-013)
%%
%%   Revision 1.1.2.4  2003/08/13 15:24:53  metze
%%   Matches release 5.0 P013
%%
%%   Revision 1.1.2.3  2002/07/31 13:09:47  metze
%%   *** empty log message ***
%%
%%   Revision 1.1.2.2  2002/07/30 13:11:11  metze
%%   *** empty log message ***
%%
%%   Revision 1.1.2.1  2002/05/28 14:12:30  metze
%%   Initial version of the Janus/ Ibis documentation
%%
%%
%% ========================================================================


\section{Installation} \label{basic:installing}

You can either just use JANUS binaries and  run Tcl-scripts or you can
check out JANUS  from CVS and access the  source code,  building JANUS
itself. How you can run JANUS depends on where you are: CMU or UKA.
If you want to compile your own version of JANUS, read on.

The JANUS binary can reside anywhere in your path, depending on the options
used during compilation, you'll need Tcl/Tk (8.0 or greater) and the
GNU readline and termcap libraries for dynamic linking.

If you use Tcl/Tk 8.4 and your Janus directory is in \$HOME/janus, you can
set your environment variables as follows (on SuSE 8.2, using \texttt{tcsh}):

\begin{verbatim}
setenv JANUS_LIBRARY        ${HOME}/janus/library
setenv TCL_LIBRARY          /usr/lib/tcl8.4
setenv TK_LIBRARY           /usr/lib/tk8.4
setenv LD_LIBRARY_PATH      ${IA32ROOT}/lib
setenv HOST                 `uname -n`
\end{verbatim}

IA32ROOT is the root directory of the Intel C++ Compiler and is only
needed if you want to compile or run programs using this compiler.
More information on settings can be found in section \Jref{file}{janusrc},
which covers the \$HOME/.janusrc configuration file for janus.

In the file \texttt{\${HOME}/.janusrc}, which is read by Janus on
startup, you should define at least the following variables:

\begin{verbatim}
set JANUSLIB       /home/metze/janus/gui-tcl
set auto_path     "/home/metze/janus/library /home/metze/janus/tcl-lib $JANUSLIB"
\end{verbatim}

Currently, janus comes in three versions:

\begin{description}
\item[Janus]   The full blown JRTk, which includes a Tcl/Tk interface and two decoders.
\item[Ibis]    The same as Janus, but without the old three-pass decoder.
\item[Decoder] A cooked-down version suitable only for decoding and without the Tcl/Tk interface.
\end{description}

All versions    usually are called   a  \texttt{janus} executable, the
``default''  version is ``Ibis'', which  does not include the obsolete
three-pass search.  If you need the old search, you'll want to look at
section \ref{basic:compiling} and compile your own janus.

%%% Local Variables: 
%%% mode: latex
%%% TeX-master: "installation"
%%% End: 
