%% <<< Skeleton generated automatically by tcl2tex.tcl

\subsection{\Jlabel{module}{CFGPTNode}}

This section describes the '\Jgloss{CFGPTNode}': \textsl{A 'CFGPTNode' object is a node of a parse tree.}

\begin{description}
\vspace{3mm}  \item[Creation:] \texttt{CFGPTNode} cannot be created directly.\

It is accessible as a sub-object of \Jref{module}{CFGParseTree}!

\vspace{3mm}  \item[Configuration:] \texttt{cfgptnode configure}


    \begin{tabular}{ll}
      \Jlabel{CFGPTNode}{-bestScore} & = 0.000000 \\
      \Jlabel{CFGPTNode}{-bestX} & = 0 \\
      \Jlabel{CFGPTNode}{-itemN} & = 1 \\
      \Jlabel{CFGPTNode}{-lvX} & = 0 \\
    \end{tabular}

\vspace{3mm} \item[Methods:] \texttt{cfgptnode}

    \begin{description}
      \Jitem{\Jlabel{CFGPTNode}{puts}} \texttt{ \Jsb{-format format}} \

        display the contents of parse tree node

      \begin{tabular}{ll}
 \texttt{\textbf{format}} &  output format (SHORT, LONG)  \\
      \end{tabular}
      \Jitem{\Jlabel{CFGPTNode}{trace}} \texttt{ \Jsb{-auxNT auxnt} \Jsb{-topN topn} \Jsb{-format format}} \

        returns parse tree by tracing back node

      \begin{tabular}{ll}
 \texttt{\textbf{auxnt}} &   print also auxilliary NTs  \\
 \texttt{\textbf{topn}} &    print the topN parse trees  \\
 \texttt{\textbf{format}} &  output format (jsgf, soup)  \\
      \end{tabular}
    \end{description}

  \item[Subobjects:] \hfill \\
\ 
    \begin{tabular}{ll}
      \texttt{\textbf{child}} & (\Jref{module}{???}) \\
      \texttt{\textbf{next}} & (\Jref{module}{???}) \\
      \texttt{\textbf{parent}} & (\Jref{module}{???}) \\
    \end{tabular}
\vspace{3mm}

  \item[Elements:] are of type \Jref{module}{CFGPTItem}.


\end{description}

%% Skeleton generated automatically by tcl2tex.tcl >>>
