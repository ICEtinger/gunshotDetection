%% <<< Skeleton generated automatically by tcl2tex.tcl

\subsection{\Jlabel{module}{CFGParseTree}}

This section describes the '\Jgloss{CFGParseTree}': \textsl{A 'CFGParseTree' object is a parse tree.}

\begin{description}
\vspace{3mm}  \item[Creation:] \texttt{CFGParseTree} cannot be created directly.\

It is accessible as a sub-object of \Jref{module}{CFG}!

\vspace{3mm}  \item[Configuration:] \texttt{cfgparsetree configure}


    \begin{tabular}{ll}
      \Jlabel{CFGParseTree}{-nodeN} & = 1 \\
    \end{tabular}

\vspace{3mm} \item[Methods:] \texttt{cfgparsetree}

    \begin{description}
      \Jitem{\Jlabel{CFGParseTree}{puts}} \texttt{ \Jsb{-format format}} \

        display the contents of parse tree

      \begin{tabular}{ll}
 \texttt{\textbf{format}} &  output format (SHORT, LONG)  \\
      \end{tabular}
      \Jitem{\Jlabel{CFGParseTree}{trace}} \texttt{ $<$spass$>$ \Jsb{-auxNT auxnt} \Jsb{-topN topn} \Jsb{-format format}} \

        returns parse tree by tracing back

      \begin{tabular}{ll}
 \texttt{\textbf{spass}} &  single pass (\Jref{module}{SPass}) \\
 \texttt{\textbf{auxnt}} &   print also auxilliary NTs  \\
 \texttt{\textbf{topn}} &    print the topN parse trees  \\
 \texttt{\textbf{format}} &  output format (jsgf, soup)  \\
      \end{tabular}
    \end{description}

  \item[Subobjects:] \hfill \\
\ 
    \begin{tabular}{ll}
      \texttt{\textbf{node(0..0)}} & (\Jref{module}{}) \\
      \texttt{\textbf{root}} & (\Jref{module}{CFGPTNode}) \\
    \end{tabular}
\vspace{3mm}

\end{description}

%% Skeleton generated automatically by tcl2tex.tcl >>>
