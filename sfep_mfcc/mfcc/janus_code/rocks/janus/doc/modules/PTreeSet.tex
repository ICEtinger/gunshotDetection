%% <<< Skeleton generated automatically by tcl2tex.tcl

\subsection{\Jlabel{module}{PTreeSet}}

This section describes the '\Jgloss{PTreeSet}': \textsl{A 'PTreeSet' object is a set of polyphone context trees.}

\begin{description}

  \item[Creation:] \texttt{PTreeSet  $<$name$>$ $<$phones$>$ $<$tags$>$ $<$modelSet$>$}


      \begin{tabular}{ll}
 \texttt{\textbf{name}} &    name of the object \\
 \texttt{\textbf{phones}} &    set of phones (\Jref{module}{Phones}) \\
 \texttt{\textbf{tags}} &      set of tags (\Jref{module}{Tags}) \\
 \texttt{\textbf{modelSet}} &  set of models \\
      \end{tabular}

\vspace{3mm}  \item[Configuration:] \texttt{ptreeset configure}


    \begin{tabular}{ll}
      \Jlabel{PTreeSet}{-blkSize} & = 100 \\
      \Jlabel{PTreeSet}{-commentChar} & = ; \\
      \Jlabel{PTreeSet}{-itemN} & = 0 \\
      \Jlabel{PTreeSet}{-name} & = ptreeSet \\
      \Jlabel{PTreeSet}{-useN} & = 1 \\
    \end{tabular}

\vspace{3mm} \item[Methods:] \texttt{ptreeset}

    \begin{description}
      \Jitem{\Jlabel{PTreeSet}{add}} \texttt{ $<$name$>$ $<$polyphone$>$} \

        adds another polyphonic tree

      \begin{tabular}{ll}
 \texttt{\textbf{name}} &       name of polyphonic tree  \\
 \texttt{\textbf{polyphone}} &  polyphone description \\
      \end{tabular}
      \Jitem{\Jlabel{PTreeSet}{index}} \texttt{ $<$names*$>$} \

        find index of a polyphone tree

      \begin{tabular}{ll}
 \texttt{\textbf{names*}} & list of names \\
      \end{tabular}
      \Jitem{\Jlabel{PTreeSet}{name}} \texttt{ $<$idx*$>$} \

        find name of a polyphone tree

      \begin{tabular}{ll}
 \texttt{\textbf{idx*}} & list of indices \\
      \end{tabular}
      \Jitem{\Jlabel{PTreeSet}{puts}} \texttt{} \

        displays the contents of a PTreeSet object

      \Jitem{\Jlabel{PTreeSet}{read}} \texttt{ $<$filename$>$} \

        reads polyphone tree from a file

      \begin{tabular}{ll}
 \texttt{\textbf{filename}} &  name of PTreeSet file  \\
      \end{tabular}
      \Jitem{\Jlabel{PTreeSet}{write}} \texttt{ $<$filename$>$ \Jsb{-minCount mincount}} \

        writes polyphone tree to a file

      \begin{tabular}{ll}
 \texttt{\textbf{filename}} &  name of tree file  \\
 \texttt{\textbf{mincount}} &   minimum count  \\
      \end{tabular}
    \end{description}

  \item[Subobjects:] \hfill \\
\ 
    \begin{tabular}{ll}
      \texttt{\textbf{list}} & (\Jref{module}{List}) \\
      \texttt{\textbf{modelSet}} & (\Jref{module}{DistribSet}) \\
    \end{tabular}
\vspace{3mm}

\end{description}

%% Skeleton generated automatically by tcl2tex.tcl >>>
