%% ========================================================================
%%  JANUS-SR   Janus Speech Recognition Toolkit
%%             ------------------------------------------------------------
%%             Object: Trouble-shooting part of documentation
%%             ------------------------------------------------------------
%%
%%  Author  :  Florian Metze & many others
%%  Module  :  troubleshooting.tex
%%  Date    :  $Id: troubleshooting.tex 2537 2004-09-24 11:39:26Z metze $
%%
%%  Remarks :
%%
%% ========================================================================
%%
%%   $Log$
%%   Revision 1.6  2004/09/24 11:38:56  metze
%%   added -expT info
%%
%%   Revision 1.5  2004/09/23 15:35:59  metze
%%   *** empty log message ***
%%
%%   Revision 1.4  2004/09/11 12:41:28  metze
%%   P014 - final?
%%
%%   Revision 1.3  2003/11/17 13:08:28  metze
%%   Version 5.0P013 fixed
%%
%%   Revision 1.2  2003/08/14 11:18:57  fuegen
%%   Merged changes on branch jtk-01-01-15-fms (jaguar -> ibis-013)
%%
%%   Revision 1.1.2.5  2003/08/13 15:49:10  metze
%%   Fixes for Hagen's taste
%%
%%   Revision 1.1.2.4  2003/08/07 13:01:25  metze
%%   *** empty log message ***
%%
%%   Revision 1.1.2.3  2003/07/14 15:30:04  metze
%%   More complete version for V5.0 P013
%%
%%   Revision 1.1.2.2  2003/07/14 13:01:08  metze
%%   *** empty log message ***
%%
%%   Revision 1.1.2.1  2003/07/14 11:16:02  metze
%%   *** empty log message ***
%%
%%
%% ========================================================================
%%
%%   COPYRIGHT BY
%%
%%   Interactive Systems Laboratories at
%%
%%        University of Karlsruhe       and     Carnegie Mellon University
%%        Dept. of Informatics                  Dept. of Computer Science
%%        Interactive Systems Labs              Interactive Systems Labs
%%        Lehrstuhl Prof.Waibel                 Alex Waibel's NN & Speech Group
%%        Am Fasanengarten 5                    5000 Forbes Ave
%%        D-76131 Karlsruhe                     Pittsburgh, PA 15213
%%        - West Germany -                      - USA -
%%
%%   Copyright (C) 1990-1995.   All rights reserved.
%%
%%   This software is part of the JANUS Speech- to Speech Translation Project
%%
%%   USAGE PERMITTED ONLY FOR MEMBERS OF THE JANUS PROJECT
%%   AT CARNEGIE MELLON UNIVERSITY OR AT UNIVERSITAET KARLSRUHE
%%   AND FOR THIRD PARTIES ONLY UNDER SEPARATE WRITTEN PERMISSION
%%   BY THE JANUS PROJECT
%%
%%   It may be copied  only  to members of the JANUS project
%%   in accordance with the explicit permission to do so
%%   and  with the  inclusion  of  the  copyright  notices.
%%
%%   This software  or  any  other duplicates thereof may
%%   not be copied or otherwise made available to any other person.
%%
%%   No title to and ownership of the software is hereby transferred.
%%
%% =================================================================================

\section{General} \label{trouble:general}

If  you don't  find the  information  you need in this  documentation,
there might       be  more    information   available      on-line  at
\htmladdnormallink{\texttt{http://isl.ira.uka.de/\~{
}jrtk/janus-doku.html}}{http://isl.ira.uka.de/~jrtk/janus-doku.html}. A
recent addition to the  JRTk  documentation is the \Jindex{Wiki}  page
available                                                           at
\htmladdnormallink{\texttt{http://www.is.cs.cmu.edu/janus/moin.cgi}}{http://www.is.cs.cmu.edu/janus/moin.cgi}.\footnote{Currently,
this                  is                       accessible           at
\htmladdnormallink{\texttt{http://penance.is.cs.cmu.edu/janus/moin.cgi}}{http://penance.is.cs.cmu.edu/janus/moin.cgi}.} 
As  this is meant to be  a ``discussion  white-board'', you might also
find help for your problem there. If you do not find an answer to your
problem    there,           please         send     e-mail          to
\htmladdnormallink{\texttt{jrtk@ira.uka.de}}{mailto:jrtk@ira.uka.de}
or directly  to one of the maintainers,  but be sure to add sufficient
debugging  information, so  that others have  a  chance  to trace  the
problem. The more (useful) information  you provide and the better you
can describe the problem, the more people will be able to help you ;-)


\section{Installation} \label{trouble:installation}

On Unix boxes,  first make sure the \texttt{janus}  binary is in  your
search  \texttt{PATH}.  If   you can't   run Janus  by   simply typing
\texttt{janus} at the shell prompt, try:

\begin{verbatim}
(i13pc33:/home/metze) setenv PATH /home/metze/janus/scr/Linux.gcc/janus:${PATH}
\end{verbatim}
% $

If you do  not use \texttt{tcsh} or  your  Janus binary is not  in the
above  directory,  you'll have    to    change nomenclature   or  path
accordingly. Janus can be compiled with or without support for X11, so
in some  cases you may  need to  set the  \texttt{DISPLAY} environment
variable:

\begin{verbatim}
(i13pc33:/home/metze) setenv DISPLAY i13pc33:0.0
\end{verbatim}

Note that e.g. \texttt{ssh -XC  i13pc33}  does not work properly  when
you redefine the \texttt{DISPLAY} environment variable, for example in
your  \texttt{.tcshrc},  if you're using this  it  is  usually best to
leave \texttt{DISPLAY} as it is.

This is the output of an interactive example trouble-shooting session
under Linux fixing several common installation difficulties:

\begin{verbatim}
i13pc33 /home/data> janus
application-specific initialization failed: no display name and no
$DISPLAY environment variable
% exit
i13pc33 /home/data> setenv DISPLAY i13pc33:0.0
i13pc33 /home/data> janus
# ======================================================================
#  ____  ____  _____ _
# |__  ||  _ \|_   _| | _     V5.1 P001 [Apr  9 2003 11:21:47]
#    | || |_| | | | | |/ |  --------------------------------------------
#    | ||  _ <  | | |   <     University of Karlsruhe, Germany
#    | ||_| |_| |_| |_|\_|    Carnegie Mellon University, USA
#   _| | JANUS Recognition
#  \__/        Toolkit        (c) 1993-2002 Interactive Systems Labs
#
# ======================================================================
application-specific initialization failed: Can't find a usable init.tcl in the
following directories:
    /home/data/janus/src/../../library /home/data/janus/src/../../../library
This probably means that JanusRTk wasn't installed properly.
% exit
i13pc33 /home/data> setenv JANUS_LIBRARY /home/data/janus/library
i13pc33 /home/data> setenv   TCL_LIBRARY /usr/lib/tcl8.3
i13pc33 /home/data> setenv    TK_LIBRARY /usr/lib/tk8.3
i13pc33 /home/data> janus
# ======================================================================
#  ____  ____  _____ _
# |__  ||  _ \|_   _| | _     V5.1 P001 [Apr  9 2003 11:21:47]
#    | || |_| | | | | |/ |  --------------------------------------------
#    | ||  _ <  | | |   <     University of Karlsruhe, Germany
#    | ||_| |_| |_| |_|\_|    Carnegie Mellon University, USA
#   _| | JANUS Recognition
#  \__/        Toolkit        (c) 1993-2002 Interactive Systems Labs
#
# ======================================================================
% puts $auto_path
/home/data/janus/library /home/data/janus/tcl-lib
/home/data/janus/gui-tcl /usr/lib/tcl8.3 /usr/lib
/home/data/janus/src/lib /usr/lib/tk8.3 /home/data/janus/library
% 
\end{verbatim}

If    you encounter one    of the   above  errors,   you can   add the
problem-solving   line to your  start-up   scripts.   The Tcl-variable
\texttt{auto\_path} can also be changed  in \texttt{.tcshrc}. As Janus
is a Tcl/Tk application,  you might also need  to install the relevant
libraries in the correct version and set  up the environment variables
\texttt{TCL\_PATH} and  \texttt{TK\_PATH}  accordingly (in  the  above
example,  the  first   of  the three    ``setenv''   lines will  often
suffice).  Some versions  of Janus  might also   be dynamically linked
against \texttt{libreadline, libtermcap,} and \texttt{libcurses}.


\section{Tcl-library problems} \label{trouble:tcllib}

Normally,   Tcl/Tk   will  automatically  source   the   files  in the
``tcl-lib'' and   ``gui-tcl'' directories,  when  functions  which are
defined in those scripts are called. If  you define new functions, you
have to  add them to  the index  file  \texttt{tclIndex}, which you'll
find in both directories. The standard way to recreate this file is to
issue  the following commands to an  instance of  janus started in the
``tcl-lib'' or the ``gui-tcl'' directory:

\begin{verbatim}
file delete tclIndex
auto_mkindex $JANUSLIB *.tcl
\end{verbatim}

%% $


\section{Object problems} \label{trouble:dbase}

Janus (more  exactly:  the Tcl/  Tk interface) uses  ``\texttt{:}'' to
access list elements  and  ``\texttt{.}'' to access  sub-objects.  For
example,  the dictionary  word    \texttt{HAS}  can be   accessed   as
\texttt{dict\$SID:HAS}    and \texttt{dict\$SID.item(100)}.  Now it is
perfectly   legal  to  have    subobjects,   whose name    contains  a
``\texttt{.}'',  although you  won't   be  able  to  access    it  via
``\texttt{:}'' now, because the ``\texttt{.}''   in the list item name
will be  interpreted  as a subobject.   The  solution is to avoid  the
``\texttt{.}''   altogether  or  to   access   your  list  item     as
\texttt{dict\$SID.item([dict\$SID.list index HAS])}.


\section{The fgets-problem} \label{trouble:fgets}

If Janus blocks (hangs) as soon  as it tries to lock  access to a file
via \texttt{fgets}, the best solution is to set  up an NGets-server by
editing the following lines in your \texttt{.janusrc} (see the example
in \texttt{\~{ }/janus/scripts/janusrc})

\begin{verbatim}
set NGETS(HOST)         ""
set NGETS(PORT)         63060
\end{verbatim}

to look somewhat like this:

\begin{verbatim}
set NGETS(HOST)         i13s8
set NGETS(PORT)         63050
\end{verbatim}

You can choose any combination of HOST and PORT you like, but the HOST
should be a reliable machine (SUNs are  great) and the PORT should not
be  used  for system services  or  somebody  else's  NGets-server. You
should now start the server on the reliable machine using

\begin{verbatim}
(i13s8:/home/metze) janus janus/tcl-lib/ngetsGUI.tcl -server
# ======================================================================
#  ____  ____  _____ _                                                  
# |__  ||  _ \|_   _| | _     V5.0 P011 [Nov 13 2002 17:17:29]
#    | || |_| | | | | |/ |  --------------------------------------------
#    | ||  _ <  | | |   <     University of Karlsruhe, Germany            
#    | ||_| |_| |_| |_|\_|    Carnegie Mellon University, USA            
#   _| | JANUS Recognition                                              
#  \__/        Toolkit        (c) 1993-2002 Interactive Systems Labs   
#                                                                       
# ======================================================================
Server accepting connection on 63060 ...
CurrentSock: sock5
\end{verbatim}

or, if you don't want the graphical interface,

\begin{verbatim}
(i13s8:/home/metze) janus
# ======================================================================
#  ____  ____  _____ _                                                  
# |__  ||  _ \|_   _| | _     V5.0 P012 [Nov 27 2002 14:43:58]
#    | || |_| | | | | |/ |  --------------------------------------------
#    | ||  _ <  | | |   <     University of Karlsruhe, Germany            
#    | ||_| |_| |_| |_|\_|    Carnegie Mellon University, USA            
#   _| | JANUS Recognition                                              
#  \__/        Toolkit        (c) 1993-2002 Interactive Systems Labs   
#                                                                       
# ======================================================================
% ngetsServerStart
Server accepting connection on 63060 ...
CurrentSock: sock5
0
%
\end{verbatim}

This NGets-server process will now  handle all calls to \texttt{fgets}
and \texttt{glob} for all other processes.  You can test this setup by
generating a simple file \texttt{/home/metze/x} containing a few lines
of text in  your home directory and  then executing Janus in your home
directory (assuming you started a server as above):

\begin{verbatim}
(i13pc33:/home/metze) echo "Line" >> x
(i13pc33:/home/metze) echo "Line two" >> x
(i13pc33:/home/metze) janus
# ======================================================================
#  ____  ____  _____ _
# |__  ||  _ \|_   _| | _     V5.0 P012 [Nov 27 2002 14:40:14]
#    | || |_| | | | | |/ |  --------------------------------------------
#    | ||  _ <  | | |   <     University of Karlsruhe, Germany
#    | ||_| |_| |_| |_|\_|    Carnegie Mellon University, USA
#   _| | JANUS Recognition
#  \__/        Toolkit        (c) 1993-2002 Interactive Systems Labs
#
# ======================================================================
INFO: Using NGETS on i13s10:63060!
% fgets x line
4
% puts $line
Line
% fgets x line
10
% puts $line
{Line two}
% fgets x line
-1
%
\end{verbatim}

Be aware that this server variant reads the file in memory once and
will only write it back when all the entries have been processed by
client processes. If your jobs die and you want to restart the jobs,
you can simply select the file and click on ``Clear'' in the
\texttt{ngetsGUI.tcl} interface window. You can check if a process is
using an NGets server by looking for the line 

\begin{verbatim}
INFO: Using NGETS on i13s10:63060!
\end{verbatim}

at startup.


\subsection*{Background} \label{trouble:fgetsbackground}

``fgets'' is an important Tcl-function, which is used in most parts of
Janus  to parallelize  jobs  on different machines.  The Janus Library
(described  in chapter \Jref{sec}{lib})  makes extensive use of it, as
do our standard testing scripts.

``fgets'' is implemented  in C (\texttt{\~{ }/janus/src/itf/itf.c}  in
case you want to have a look).  If you run  JANUS on a single machine,
using

\begin{verbatim}
while {[fgets spkList spk] != -1} {
    puts $spk
}
\end{verbatim}

is equivalent to 

\begin{verbatim}
set fp [open spkList r]
while {[gets $fp line] != -1} {
    puts $spk
}
close $fp
\end{verbatim}

Both   scripts will  print    out    the   contents  of  the      file
\texttt{spkList}.   If, however you run   the same script on different
machines on  the same file  and at the same  time, you will notice the
difference:  The first version   will ``divide'' the list between  the
different machines, while the second version will print the whole list
on every machine. Also, if you  have a look at the  file after you ran
the first  script, you will notice that  the  first character of every
line is no ``\#''.  Running this script on such a file will produce no
output, because it ``believes'' that all ``keys'' (lines) have already
been  processed  (output) by another machine.  It  is therefore a good
idea to keep backup copies of speaker lists etc. around.

On some  machines or operating  systems  (e.g. Linux  with certain nfs
implementations),  this  mechanism  does not    work reliably, because
exclusive file  locking  cannot be guaranteed,  e.g.  two machines can
read and write to one file at the same time.   The easiest solution to
this problem   is to re-define  ``fgets''  in   Tcl  and replace  this
mechanism  by  something  else,  i.e. a server   that reads  files and
listens   on ports.  Such an  approach   is implemented in 
\texttt{\~{ }/janus/tcl-lib/ngets.tcl},                and             
\texttt{\~{ }/janus/tcl-lib/ngetsGUI.tcl}.


\section{Catching aborts} \label{trouble:abort}

Janus  is implemented  in C.   Some program  faults will  therefore be
caused by segmentation violations. C has handlers to catch a seg-fault
signal and  execute specific code.  The   relevant procedure is called
\texttt{\~{ }/janus/src/itf/itc.c:janusSignalHandler} and can be used
to send  mail  or do   something  else if    you  define a   procedure
``janusErrorHandler'' at Tcl-level.

Code like   this (in combination with  other  approaches)  can be very
useful in  maximising CPU load  during evaluation times, while it will
not improve the quality of  the code. If  you get  aborts, it will  be
best to debug the code .

An example procedure that will send e-mail if Janus crashes inexpectedly
looks like this:

\begin{verbatim}
proc janusErrorHandler { sig } {
    global errorInfo errorCode argv argv0 env

    set sigN [lindex "NONE SIGHUP SIGINT SIGQUIT SIGILL 5 SIGABRT 7 \
                      SIGFPE SIGKILL 10 SIGSEGV 12 SIGPIPE SIGALRM SIGTERM" $sig]
    regsub   "\\..*" [info hostname] "" host
    set exe  [info nameofexecutable]
    set cmd  "$argv0 $argv"
    set pwd  $env(PWD)
    set mail $env(USER)@ira.uka.de

    switch $sigN {
        SIGABRT -
        SIGFPE  -
        SIGSEGV {
            janusSendMail $mail \
                "$sigN $host $pwd: $argv0" \
                "$host.[pid] $pwd:\n$exe $cmd\n[string repeat - 72]\n$errorInfo"
        }
        default {
            puts stderr "\nReceived signal $sig ([lindex $sigL $sig]).\n"
        }
    }
}

proc janusSendMail { address subject body } {
    exec echo $body | mailx -s $subject $address
}
\end{verbatim}
%% $

Define these procedures in your \texttt{.janusrc} and you'll receive
e-mails when janus seg-faults. A system to notify the user of all
possible errors is however difficult to realise :-(.


\section{Filesystem issues} \label{trouble:filesys}

Janus can   read compressed files  transparently.   In some cases  the
piping  mechanism  used however  causes  problems,  so  if you see  an
I/O-related problem on compressed files, try working with uncompressed
(or local) files first.

Accumulating and particularly combining   ML accumulators can pose   a
heavy burden on distributed filesystems. If you  want to guarantee the
execution of the  server part of the doParallel  loop  on a particular
machine (i13pc44 in this case), for example because this machine holds
the data  locally  or has   a very  fast network connection,   you can
include the following code in your \texttt{.janusrc}:

\begin{verbatim}
proc doParallelServer { } {
    set SERVER [lindex [glob -nocomplain "i13pc44.*.INIT"] 0]
    if {$SERVER == ""} {
        set SERVER [lindex [glob -nocomplain "i13pc4\[0-6\].*.INIT"] 0]
    }
    if {$SERVER == ""} {
        set SERVER [lindex [lsort -decreasing [glob -nocomplain "i13pc3*.*.INIT"]] 0]
    }
    if {$SERVER == ""} {
        set SERVER [lindex [glob -nocomplain "i13pc5\[0-1\].*.INIT"] 0]
    }
    if {$SERVER == ""} {
        set SERVER [lindex [glob -nocomplain "i13pc2*.*.INIT"] 0]
    }
    if {$SERVER == ""} {
        set SERVER [lindex [lsort [glob "*.INIT"]] end]
    }
    return [string range $SERVER 0 [expr [string length $SERVER]-6]]
}
\end{verbatim}
% $

If i13pc44 is not available,  this procedure will choose the next-best
machine and so on.


\section{featAccess and featDesc} \label{trouble:feat}

The  \Jref{file}{featAccess}  and \Jref{file}{featDesc}  file serve to
define where to find acoustic data and how to  process it. They are in
fact Tcl scripts  evaluated in a separate  interpreter.  The reason to
hold them separately is to  allow for greater flexibility when porting
systems between tasks, architectures, or sites.

The fact that these scripts are evaluated as Tcl-scripts in a separate
interpreter limits the   scope  of variables;  if you're  experiencing
error messages stemming from  featAccess or featDesc, debugging can be
a bit tedious, because you   cannot run the scripts interactively  and
determine which variables are visible  or which commands fail (and for
what reason).

\begin{verbatim}
% featureSet$SID eval $uttInfo
  warp /project/MT/data/ESST/cd28/e044a/e044ach2_039.16.shn with factor 
1.000
ERROR   matrix.c(2080)     expected 'float' type elements.
ERROR   itf.c(0359) <ITF,FCO>  Failed to create 'dummyS' object.
ERROR   itf.c(0720)        featureSetEval<featureSetQ4g> featureSetQ4g 
{{spk MBB_e044ach2} {utt e044ach2_039_MBB} MBB_e044ach2 {EDUCATION 
graduate} {PROFESSION student} {NATIVE_LANG e} {SEX m} {ID MBB} {KEY 
MBB_e044ach2} {DIALECT American English} {DATE_OF_BIRTH 710808} 
{PRIMARY_SCHOOL Louisville, KY} {SEGS e044ach2_001_MBB e044ach2_003_MBB 
e044ach2_150_MBB e044ach2_152_MBB} e044ach2_039_MBB {ADC e044ach2_039.16} 
{ID MBB} {LM yeah #NIB_H## #NIB_H## though it is #NIB_GE# what #NIB_UM# 
seven hours #NIB_GE#} {TEXT yeah #NIB_H## #NIB_H## though it is #NIB_GE# 
what #NIB_UM# seven hours #NIB_GE#} {CHANNEL e044ach2} {PATH cd28/e044a} 
{KEY e044ach2_039_MBB} {TIME 4.181} {SPEAKER MBB_e044ach2}}:
\end{verbatim}

In this  example, you can determine that  the error occured during the
evaluation       of   featDesc  (\texttt{featureSetEval<featureSetQ4g>
featureSetQ4g}); the    exact kind and   location  of  the  error (the
subtraction  of  spectral means failed  because  none  were loaded) is
usually determined by the insertion of  several \texttt{puts ``Now I'm
here    ...''}  and    \texttt{puts  ``WARPFACTOR=\$WARP''} lines   in
\Jref{file}{featDesc}.


\section{Score functions} \label{trouble:scoreA}

Janus spends most of its time doing score computations. This is
usually done in a highly optimized routine called \texttt{ssa\_opt}.
This routine makes a few assumptions makes a few assumptions on
the underlying acoustic models, namely:

\begin{itemize}
\item They are fully continuous
\item All codebooks share the same feature
\end{itemize}

So, if you use multiple STC classes or you're building a semi-continuous
system, be sure to add the following line to your decoding script:

\begin{quote}
\texttt{senoneSet\$SID} \Jref{SenoneSet}{setScoreFct} \texttt{base}
\end{quote}

Also, if your acoustic scores look really bad, you might try this
line, too.


\section{Labels and Dictionaries} \label{trouble:labeldict}

Labels store  pre-computed time-alignments as  computed by the Viterbi
or Forward-Backward  algorithm.  If you're  using labels  and  you get
error messages stating

\begin{verbatim}
Couldn't map 234 of 1234 path items.
\end{verbatim}

or  the  results     from training  are   unreasonable,   usually your
\Jref{module}{Path}    (labels)   and your  current \Jref{module}{HMM}
construction don't  match. Labels store  state indices, i.e. ``frame X
occupies the  HMM state(s) Y (and  Z)''. If the HMM  object associated
with the Path  object when  reading the  labels was  built differently
from the one used during label writing, the indexing will be different
(i.e. skewed in time) and the  labels are essentially useless. Typical
culprits changed during HMM construction are:

\begin{itemize}
\item A modified \Jref{module}{Dictionary}
\item The \texttt{-optWord} and \texttt{-variants} flags to HMM \Jref{HMM}{make}
\item Different transcriptions (filtered differently, more pauses, etc.)
\end{itemize}

If you want  to  change any of the   above, your time alignments  will
change anyway, so you'll need to write new labels. In some cases it is
possible to  re-use old (Viterbi) labels by  creating  the old and new
HMMs side by side  and re-configuring the path items' \texttt{-stateX}
by hand  (you'll have to  create  them by \Jref{Path}{bload}  or  some
other method),  but you better know exactly  what you're doing or your
results will be bogus.


\section{Language Models} \label{trouble:lm}

Janus can read in ARPABO-format language models. However, depending on
the sorting order  of  n-grams, this can  take  a long time or  simply
fail. In this case,  try to re-sort the file  so that the first column
is sorted  first and/ or try   to load the  uncompressed  file. If you
experience  trouble reading a compressed  LM file (as  in dump files),
read on with section \ref{trouble:dumps}.


\section{Defines and Dump Files} \label{trouble:dumps}

The  Ibis decoder  stores  files in  a   compressed binary format  for
reasons of I/O speed and memory consumption on disk. Particularly, the
STree    \Jref{STree}{dump} contains   a    \Jref{module}{Dictionary},
\Jref{module}{SenoneSet},                       \Jref{module}{SVocab},
\Jref{module}{PHMMSet},  \Jref{module}{LCMSet}, \Jref{module}{RCMSet},
and      possibly  an    \Jref{module}{XCMSet}      as  well  as    an
\Jref{module}{LingKS}/ \Jref{module}{SVMap}.

If  you're trying to   read a dump file  and  you're getting  an error
message, chances are the  system the  dump was  written with  a system
which differs from  the system you're reading  the  dump with  in some
aspect. Favourites are:

\begin{description}

\item[Objects:] used  in one system, but   not the other, particularly
the  \Jref{module}{XCMSet}. Also,  when  loading  a dump,  the  object
reading the dump has  to be empty, i.e. it   has to exist, but  cannot
contain data.

\item[Configuration:]  some objects   are  configured differently   or
contain a different number of items.

\item[Language Models:]   the  \Jref{module}{MetaLM} relies  on other,
lower-level models and  only  stores the top-level information  in its
dump file; therefore the  lower LMs  have  to be  initialized properly
before reading a dump into a \Jref{module}{MetaLM}.

\item[Executable:]  Janus and    the Ibis decoder  depend  on    a few
compile-time   \texttt{\#define}s  and    \texttt{typedef}s   set   in
\texttt{src/ibis/slimits.h}.  See section \ref{basic:compiling}    for
information on how to compile Janus and the meaning of these flags. If
you have  two executables compiled  with different settings,  you will
most likely  not be   able to exchange  dump-files between    the two,
because the internal representation of data is different.

\end{description}


\section{Speed} \label{trouble:speed}

If you find your training or decoding  is taking too long, chances are
you're  working  on an ill-conditioned problem,    e.g. you'll have to
think about a more intelligent setup.

The first step  in  finding out why your    jobs run so slowly  is  to
pinpoint the     part of  the   code   which takes   up   most  of the
time. Frequently, your job  is just to big (takes  up too much memory)
and the machine  starts ``swapping''. If this is  the case, try moving
to another machine or look at section \ref{trouble:memory}. If this is
not the case, usual suspects are:

\begin{description}

\item[I/O:] it can take a long time to  load data into memory, if it's
stored on network disks or  is distributed over  many small files. Try
to move the  data to a local disk  or reduce the  number of individual
files by  using  a differently organized  \Jref{module}{DBase}, a more
intelligent \Jref{file}{featDesc}  and \Jref{file}{featAccess}  or  by
using a \Jref{module}{Labelbox} (c.f.  sections \ref{trouble:feat} and
\ref{trouble:filesys}).

\item[Decoding:] Try  reducing the  beams used in \Jref{module}{SPass}
or    use     a     different     scoring    function    by      using
\Jref{SenoneSet}{setScoreFct}. Popular speed-up techniques include the
use of a  \Jref{module}{BBITree}  and \Jref{SPass}{fmatch}.  Sometimes
it is necessary  to  use profiling to  see  where exactly the  time is
spent.

If      you're   using    the      ``base''     scoring       function
(\Jref{SenoneSet}{setScoreFct}),  try      to configure   the Gaussian
evaluation threshold  differently,  i.e. set \Jref{Cbcfg}{-expT} to  a
higher cutoff such as in

\begin{quote}
\texttt{codebookSet\$SID:SIL-m.cfg configure -expT -10.0}.
\end{quote}

This  should  speed up your  decoding by  up  to 20\%  without loss in
accuracy. Also see section \ref{trouble:scoreA} on scoring functions.

\item[Language Models:] Complicated  language  models can take a  long
time to evaluate, try using a simpler language model for decoding, use
a  simple  n-gram  \Jref{module}{LingKS} as  look-ahead  by using  the
\texttt{-map} option when  creating  an \Jref{module}{LTree},  and/ or
play  with   the \texttt{-cache}    and \texttt{-mode}   settings   in
\Jref{module}{LTree}.

\end{description}

You can also check the compile-time settings  of Janus. You might find
a setting which optimizes speed for your particular hardware.


\section{Memory Consumption} \label{trouble:memory}

If you find  your training or  decoding uses too much  memory, chances
are you're working on an  ill-conditioned problem, e.g. you'll have to
think about a more intelligent setup.

You  can reduce the memory   footprint of the executable by  compiling
Janus as ``Ibis''   only (c.f.  section \ref{basic:compiling}) or   by
using different   \texttt{\#define}s   and \texttt{typeset}s   for the
\texttt{LVX} and \texttt{SVX} types used in  the decoder (see sections
\ref{basic:compiling} and \ref{trouble:dumps}).


%%% Local Variables: 
%%% mode: latex
%%% TeX-master: t
%%% End: 
